% peldak.tex - Latex példák.
%
% A túlnyujtott rizsázásokat inkább lipsummal helyettesítsd.
%
% ===================== BimbaLaszlo(.co.nr|gmail.com) ==== 2014.02.03 05:18 ==

\documentclass[a4paper]{report}             % book: alapból dupla-oldalas
                                            % report: nem ismeri a 'part' egységet
                                            % article: nem ismeri a 'part' és 'chapter' egységeket
\usepackage[utf8]{inputenc}                 % A szöveg UTF-8 karikódolású.
\usepackage[magyar]{babel}                  % Magyar címek és elválasztások.
\usepackage[T1]{fontenc}                    % Valószínűleg betűtípus beállítás
                                            % (mellőzhető)
\PassOptionsToPackage{defaults=hu-min,labelenums=hu-A}{magyar.ldf}

\usepackage{fullpage}                       % 1inch margó, a szöveg szélessége (\textwidth) 159,2mm.
\usepackage{makeidx}                        % Csináljon indexet. ( szójegyzék )
\makeindex

% EZZEL VALAMI NAGYON NEM STIMMEL!
\usepackage{hyperref}                       % Hiperhivatkozások használata.
\hypersetup{
    unicode=true,
    colorlinks=true,
    linkcolor=red
}

\usepackage{listings}                       % A listings típusú forráskódokhoz kell.
\usepackage{textcomp}                       % A listings upquote beállításához.
% Listings megjelenítéseinek beállításai.
\lstset{showtabs=true}                      % Tabokat nyíllal jelenítse meg. (makefile kód estén pl.)
\lstset{tab=\rightarrowfill}                % A tab egy nyíllal legyen jelölve.
\lstset{tabsize=4}                          % 1 tab = 4 szóköz
\lstset{showstringspaces=false}             % Idézőjeles szövegben ne jelölje a szóközt.
\lstset{upquote=true}                       % A ` és ' kari valódi formájában jelenjen meg.
\lstset{frameround=tftf}
\lstset{frame=trBL}                         % A kódot körülvevő keret.
% Értékei lehetnek:
% single Kerettel veszi körbe a kódot.
% t      Vonal beszúrása felülre ...
% b     ... alulra ...
% r     ... jobbra ...
% l     ... balra ...
% T     ... felülre, de dupla vonal.
% A nagybetű mindenhol dupla vonalat eredményez.

\usepackage{graphicx}                       % Képek beillesztéséhez.

% _________________________________ BEGIN ____________________________________

\title{  \LaTeX\ PÉLDÁK }                   % A mű címe ...
\author{ Bimba László }                     % ... szerzője ...
\date{   2011. február 19. -- \today }      % ... és a keletkezés dátuma.

\begin{document}    % Elkezdjük a dokumentumot.

\maketitle          % Cím, szerző, stb. kiírása.
\tableofcontents    % Tartalomjegyzék kiírása. ( automatikusan generálja )

% Az 'article' stílus ezeket nem ismeri:
\part{Példák}
\chapter{\dots még mindig}

Egyszerű, de szemléletes példák. Az első bekezdésének első sora nincs behúzva.
Ha egy szöveget (a \LaTeX\ kódban) új sorba írunk, az semmilyen látható
változást nem eredményez. Hogy látható legyen a kész dokumentumban is, azt a
\verb"\\" jelsorozat beszúrásával érhetjük el -- ez a soremelés. \\
Így a szöveg valóban új sorba kerül.

Ha két sor között hagyunk egy üreset, azzal új bekezdést indítunk el. Ha azt
szeretnénk, hogy ne csak a kóban legyen térköz, hanem a kész dokumentumban is,
akkor két soremelést kell végeznünk, de így továbbra is ugyanabban a
bekezdésben maradunk. \\
\\
Ahhoz, hogy két -- valóban külön álló -- bekezdés közé is beszúrjunk egy üres
sort, tegyük az első bekezdés végére az üres sor \verb"\\" jelét és hagyjunk
egy üres sort a kódban, hogy új bekezdés kezdődjön.
\\

Ha a térköz nagyságát szeretnénk megadni, akkor a \verb"\\[méret]" soremelést
használjuk, például most következzen egy 3 betű magasságú szünet:
\verb"\\[3em]" \\[3em]

A szöveg lehet \textrm{roman}, \textsf{sans serif}, \texttt{typewriter},
\textmd{medium}, \textbf{bold}, \textup{upright}, \textit{italic},
\textsl{slanted}, \textsc{small caps}, \emph{emphasized} és
\underline{underline} alakú.
\\

Ha egy oldalon már nem kívánunk további szöveget megjeleníteni, akkor
használjuk a \verb"\pagebreak" parancsot.

\pagebreak

\section{Főcím}
\section*{Számozatlan főcím}
\subsection{Alcím}
\subsubsection{Legalja}

\paragraph{Paragrafus}

Bár az eredeti rendeltetését nem tudom, de definíció leírásához használhatjuk
például, mivel a cím és az őt követő szöveg egy sorban kezdődik.

\subparagraph{Alparagrafus}

Behúzással különül el a fő paragrafustól, viszont ha több soron keresztül
tart, a második és az azt követő soroknak nincs behúzása.

\begin{quote}

    Ez egy idézet (\verb"quot") blokk, minden sora be van húzva az őt környező
    szöveghez képest, így külön el a fő szövegtől.

\end{quote}

Emez egy \verb"description":

\begin{description}
    \item[Definiálandó szöveg] Magyarázó szöveg.
    \begin{description}
        \item[Ezen belül] található ez.
    \end{description}
\end{description}

% ________________________________ TÁBLÁZAT __________________________________

\section{Táblázat}          % Főcím.
\index{tabular}             % Tedd indexbe ezen oldal számát az adott szóhoz.

\begin{table}[ht]           % h: here, b: bottom of page, t: top of page

    \centering
    \begin{tabular}{||c|lr|}

        \hline
        gnats       & gram      & \$13.65         \\
        \cline{2-3}
                    & each      & .123            \\
        \hline
        gnu         & stuffed   & 98.50           \\
        \cline{1-1} \cline{3-3}
        emu         &           & 33.33           \\
        \hline
        armadillo   & frozen    & 8.99            \\
        \hline
        lorem       & \multicolumn{2}{|c|}{ipsum} \\
        \hline

    \end{tabular}

    \caption{Példa}         % Magyarázószöveg.
    \label{tbl-pelda}       % Hivatkozás a táblázatra.

\end{table}

Szép, logikusan felépíthető -- lásd a forrásban. Ha a \verb"tabular" részt
használjuk a \verb"table" nélkül, úgy nem lesz térköz a táblázat körül.

% _______________________________ FORRÁSKÓD __________________________________

\section{Forráskód}

Ha a szövegkörnyezetben egy parancsót ki akarunk emelni, akkor tegyük a
\verb!\verb"parancs"! jelek közé. (a \verb!"! jel helyén egyéb karakter is
állhat, mint pl. a \verb"!", vagy \verb"$" jel)

% ................................ VERBATIM ..................................

\subsection{Verbatim}
\index{verbatim}

Gyakorlatilag csak a betűtípust változtatja meg, a tabulátort nem ismeri, az
ékezeteket hülyén kezeli.

% center?
Verbatim példa (alapból nem lehet \verb"caption" (képaláírás) tulajdonsága)

\begin{verbatim}
/* Árvíztűrőtükörfúrógép. */

#include <stdio.h>

int
main( void )
{
    printf( "Hello Kitty!" );

	/* Tabulatoros sor, ` es ' jel megjelenitese. */

    return( 0 );
}
\end{verbatim}

% ................................ LISTINGS ..................................

\subsection{Listings}
\index{lstlisting}

Sok dolgot be lehet állítani, de az ékezeteket például alapból nem is
értelmezi. (UTF-8 esetén biztosan)

\begin{lstlisting}[language=C,caption={Listings példa},label=lst_pelda]
#include <stdio.h>

int
main( void )
{
    printf( "Hello Kitty!" );

	/* Tabulatoros sor, ` es ' jel megjelenitese. */

    return( 0 );
}
\end{lstlisting}

% _______________________________ MATEMATIKA _________________________________

\section{Matematika}

Ha egy matematikai képletet akarunk beszúrni a szövegbe, akkor a \verb"$"
jellel kell körül venni azt -- például a $\frac{3}{4} \times \tan \alpha$
képletet. Ha a képletet külön sorban szeretnénk megjeleníteni, akkor
használhatjuk a \verb"\[ 3 + 2 \]" formát \dots
\[ 3 + 2 \]
\dots vagy a \verb!\begin{equation}! blokkot is, így számozott képleteket
kapunk.

\begin{equation}
    \sin \alpha + 2 = \cos ( \alpha + 90 ^{\circ} ) + \sqrt[3]{8}
\end{equation}
\begin{equation}
    \sum_{k=2}^n k^2 = 2^2 + 3^2 + 4^2 + \dots + n^2
\end{equation}
\begin{equation}
    x! = x_1 + x_2 + x_3 + \dots + x_{n-1} + x_{n}
\end{equation}

% ________________________________ GRAFIKA ___________________________________

\section{Grafika}

A \LaTeX\ alapjába véve az \verb".eps" képformátumot ismeri csak. Használhatjuk
a MikTeX saját konvertáló programját is, hogy egy \verb".png", vagy
\verb".jpg" képet átalakítsunk a kívánt formára:

\begin{verbatim}
bmeps.exe -c kep.png latex_kep.eps
\end{verbatim}

Ennek ellenére a MikTeX \verb"graphicx" csomagja épphogy az \verb".eps"
kiterjesztést nem ismeri, ellenben a \verb".png" formátumot elfogadja.\\
\\
Sajnos eléggé korlátozottak a képmódosító képességei, így jobb, ha előbb
magunk méretezzük át a képet, hogy ne romoljon a végeredmény minősége.\\
\\
\begin{em}
Lorem ipsum dolor sit amet, consectetur adipiscing elit. Morbi vel libero sed
augue hendrerit aliquam ut ultricies sapien. Aliquam condimentum viverra eros
eget suscipit. Curabitur magna dolor, pretium non adipiscing sit amet, commodo
nec sem. Vivamus lobortis dolor vitae mi malesuada pulvinar. Nam a rhoncus
augue. Aliquam in malesuada felis. Nam faucibus iaculis enim ut iaculis. Nulla
molestie enim nec tortor mollis elementum. Praesent fermentum enim lacinia
ante suscipit sollicitudin. Quisque facilisis tortor et lorem gravida
tristique eget ac tellus. Cras non urna at mauris euismod hendrerit et a diam.
Integer sagittis facilisis nunc quis lobortis. Donec et leo quis est
pellentesque accumsan. Nulla vel orci sit amet lorem dapibus feugiat vel non
ipsum. Quisque egestas imperdiet diam vel ullamcorper. Nulla luctus sapien vel
leo euismod in mattis nunc ultrices. Cum sociis natoque penatibus et magnis
dis parturient montes, nascetur ridiculus mus. Nullam id ligula nunc. Maecenas
tincidunt feugiat leo eu metus. \\
Lorem ipsum dolor sit amet, consectetur adipiscing elit. Morbi vel libero sed
augue hendrerit aliquam ut ultricies sapien. Aliquam condimentum viverra eros
eget suscipit. Curabitur magna dolor, pretium non adipiscing sit amet, commodo
nec sem. Vivamus lobortis dolor vitae mi malesuada pulvinar. Nam a rhoncus
augue. Aliquam in malesuada felis. Nam faucibus iaculis enim ut iaculis. Nulla
molestie enim nec tortor mollis elementum. Praesent fermentum enim lacinia
ante suscipit sollicitudin. Quisque facilisis tortor et lorem gravida
tristique eget ac tellus. Cras non urna at mauris euismod hendrerit et a diam.
Integer sagittis facilisis nunc quis lobortis. Donec et leo quis est
pellentesque accumsan. Nulla vel orci sit amet lorem dapibus feugiat vel non
ipsum. Quisque egestas imperdiet diam vel ullamcorper. Nulla luctus sapien vel
leo euismod in mattis nunc ultrices. Cum sociis natoque penatibus et magnis
dis parturient montes, nascetur ridiculus mus. Nullam id ligula nunc. Maecenas
tincidunt feugiat leo eu metus.
\end{em}

\begin{figure}[ht]
    \caption{grafika.png}
    \includegraphics[width=\textwidth]{grafika.png}
\end{figure}

% ________________________________ FÜGGELÉK __________________________________

\appendix                                   % Innentől függelékek következnek.

% Alapból nem írja ki a címet és nem is rakja tartalomjegyzékbe.
% Megoldás a 'report' és 'book' dokumentumstílus esetén:
\addtocontents{toc}{\bigskip \textbf{Függelék} \\ }

% 'article':
%\section*{Függelék}
%\addcontentsline{toc}{section}{Függelék}

\listoftables                               % Táblázatok jegyzéke.
Táblázat caption-re ránézni!

\listoffigures                              % Ábrák jegyzéke.

\printindex                                 % Tárgymutató.

\end{document}
